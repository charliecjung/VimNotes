\documentclass[12pt, letterpaper]{article}
\usepackage[utf8]{inputenc}
\usepackage{enumitem}
\usepackage{dirtytalk}
%Preamble:
\title {\textbf{Vim} Notes}
\author{Charlie Jung}
\date{\today}

\begin{document}
\maketitle

\newpage
\underline{Vim:} 
\\


\begin{flushleft}



\textbf{h} = \leftarrow 
\par

\textbf{k} = \uparrow 
\par

\textbf{l} = \rightarrow 
\par

\textbf{j} = \downarrow 
\par

\textbf{x} = delete letter \\
\par 
\textbf{i} = insert ch. at cursor  \\
\par
\textbf{A} = append text \\
\par
\textbf{:wq} = quit and \textit{save} a file \\
\par
\textbf{:q!} = quit without \underline{saving} \\
\par
\textbf{dw} = delete a word \\
\par
\textbf{d\$} = delete until the end of the line \\
\par
\textbf{d} = delete operator + <motion> 

\begin{itemize}[leftmargin = 10ex]
    \item \textbf{w} = until the start of the next word (\textit{Excluding} the first character) 
    \item \textbf{e} = to the end of the current word (\textit{Including} the last word) \par
    \item \textbf{\$} = to the end of the line (\textit{Including} the last character)  \par 
    \item \textbf{d} = delete the whole line \par 
\end{itemize}


\underline{Pressing just the motion keys will move to that position} \\

A \textbf{number} before the action will repeat it that many times. \\

For example, \\

\textbf{0 (\#)} = moves to the start of the line \\
\par

\textbf{<d> <number> <motion>} \\
\par

\textbf{dd} = delete whole line \\
\par

\textbf{u} = undo last command \\
\par

\textbf{U} = fix whole line \\
\par

\textbf{p} = put deleted text \\
\par

\textbf{r<Letter>} = replace letter at cursor with \textbf{<Letter>} \\
\par

\textbf{ce} = change until the end of the \underline{word} \\
\par

\textbf{c <number> <motion> } \\
\par

\textbf{CTRL + G} = \say{Display File} and list line number etc... \\
\par

\textbf{G} = to move to a certain line \# in a file \\
\par
or \\
\par
\textbf{<\#>G} \\
\par

\textbf{gg} = Move to the \textbf{START OF THE FILE} \\
\par

\textbf{G} = Move to the \textbf{END OF THE FILE} \\
\par

\textbf{/<PHRASE>} = To search for the \underline{\underline{specified phrase}}. 
\par
To search for the same phrase again: \textbf{n} 
\par
Opposite direction: \textbf{N} \\
\par

To search in the backward direction: \textbf{?} 
\par
To go back = \textbf{CTRL + O} \Leftarrow repeat \hspace{1ex} to \hspace{1ex} go \hspace{1ex} further 
\par
\textbf{CTRL + I} = goes forward \\
\par

\textbf{\%} = find matching ), ], \} \par
Press \% to switch to the staring/ending symbol \\
\par

\textbf{:s/old/new/g} = substitute "old" with "new". \par
(changes "old" \rightarrow "new" for \underline{current line}) \par
/g = current line \\
\par

\textbf{:#, #s/old/new/g} = #, # are line numbers \\
\par

\textbf{:\%s/old/new/g} = change every occurrence of "old" to "new" in the \textit{whole file} \\
\par

\textbf{:\%s/old/new/gc} = change every occurrence in the whole file \underline(with a prompt) \\
\par

\textbf{:!, :!ls, :!dir} = followed by \underline{ANY} external command \\
\par
\textbf{:!ls, ~dir} = list current directory \\
\par

\textbf{:w <FILENAME?} = save a new file with the name as <FILENAME> \\
\par

\underline{MS-DOS} = \textbf{!del TEST} \par
\underline{UNIX} = \textbf{!rm TEST} \\
\par

\textbf{v} = highlight text and enter Visual Mode \\
\par
E.g. Man! \frame{Oh man!} \\
\par
\textbf{d} = delete \\
\par
\textbf{:w Bam} = saves highlighted text into a new file: Bam \\
\par

\textbf{:r <FILENAME>} = insert contents of a file. (Can also pipe out \underline{file output}) \\
\par

\textbf{:r !ls} = prints out the current file directory \\
\par

\textbf{o} = open a line \textit{under} the cursor and places you in INSERT mode \\
\par

\textbf{O} = open a line \textit{above} the cursor and places you in INSERT mode \\
\par

\textbf{a} = append text after the cursor \\
\par

\textbf{R} = replace more than one \underline{character} \\
\par

R 123 \rightarrow 479

\textbf{y} = copy \\
\par

\textbf{p} = paste \\
\par

\section{Miscellaneous Tips}

You can \underline{\underline{ignore case}}: \\
\par

:set ic \\
\par

highlight \\
\par

:set hls \\
\par

:set incsearch \\
\par

nohls = to disable highlight \\
\par

ignore case for one search command: \par
\c or /ignore\c \\
\par

\frame{fn + F1 or :help} \\
\par

\underline{Vim StartUp Scripts with .vimrc} \\
\par

\underline{MS-WINDOWS}: e \$VIM/_vimrc \\
\par

\underline{UNIX}: e ~/.vimrc \\
\par
Read examples:

\textbf{:r \$VIMRUNTIME/vimrc_example.vim} \\
\par

\underline{Command-Line Completion}: \\
\par

\begin{enumerate}
    \item First, make sure that vim is not in compatible mode \par \textbf{:set nocp} \\ \par
    \item See what files exist in the \underline{current directory}: \par \textbf{!ls or :!dir} \\ \par
    \item Type the start of the command: \textbf{:e} \\ \par \textbf{CTRL + D} \rightarrow to \hspace{1ex} show \hspace{1ex}  available \hspace{1ex}  commands \\ \par \textbf{<TAB>} \rightarrow to \hspace{1ex}  autocomplete \hspace{1ex}  a \hspace{1ex}  command \\ \par
\end{enumerate}




 























\end{flushleft}
\end{document}